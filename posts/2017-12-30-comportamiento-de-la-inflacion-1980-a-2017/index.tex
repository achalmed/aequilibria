\documentclass[
  stu,
  floatsintext,
  longtable,
  a4paper,
  nolmodern,
  notxfonts,
  notimes,
  12pt,
  colorlinks=true,linkcolor=blue,citecolor=blue,urlcolor=blue]{apa7}

\usepackage{amsmath}
\usepackage{amssymb}



\usepackage[bidi=default]{babel}
\babelprovide[main,import]{spanish}


% get rid of language-specific shorthands (see #6817):
\let\LanguageShortHands\languageshorthands
\def\languageshorthands#1{}

\RequirePackage{longtable}
\RequirePackage{threeparttablex}

\makeatletter
\renewcommand{\paragraph}{\@startsection{paragraph}{4}{\parindent}%
	{0\baselineskip \@plus 0.2ex \@minus 0.2ex}%
	{-.5em}%
	{\normalfont\normalsize\bfseries\typesectitle}}

\renewcommand{\subparagraph}[1]{\@startsection{subparagraph}{5}{0.5em}%
	{0\baselineskip \@plus 0.2ex \@minus 0.2ex}%
	{-\z@\relax}%
	{\normalfont\normalsize\bfseries\itshape\hspace{\parindent}{#1}\textit{\addperi}}{\relax}}
\makeatother




\usepackage{longtable, booktabs, multirow, multicol, colortbl, hhline, caption, array, float, xpatch}
\usepackage{subcaption}


\renewcommand\thesubfigure{\Alph{subfigure}}
\setcounter{topnumber}{2}
\setcounter{bottomnumber}{2}
\setcounter{totalnumber}{4}
\renewcommand{\topfraction}{0.85}
\renewcommand{\bottomfraction}{0.85}
\renewcommand{\textfraction}{0.15}
\renewcommand{\floatpagefraction}{0.7}

\usepackage{tcolorbox}
\tcbuselibrary{listings,theorems, breakable, skins}
\usepackage{fontawesome5}

\definecolor{quarto-callout-color}{HTML}{909090}
\definecolor{quarto-callout-note-color}{HTML}{0758E5}
\definecolor{quarto-callout-important-color}{HTML}{CC1914}
\definecolor{quarto-callout-warning-color}{HTML}{EB9113}
\definecolor{quarto-callout-tip-color}{HTML}{00A047}
\definecolor{quarto-callout-caution-color}{HTML}{FC5300}
\definecolor{quarto-callout-color-frame}{HTML}{ACACAC}
\definecolor{quarto-callout-note-color-frame}{HTML}{4582EC}
\definecolor{quarto-callout-important-color-frame}{HTML}{D9534F}
\definecolor{quarto-callout-warning-color-frame}{HTML}{F0AD4E}
\definecolor{quarto-callout-tip-color-frame}{HTML}{02B875}
\definecolor{quarto-callout-caution-color-frame}{HTML}{FD7E14}

%\newlength\Oldarrayrulewidth
%\newlength\Oldtabcolsep


\usepackage{hyperref}




\providecommand{\tightlist}{%
  \setlength{\itemsep}{0pt}\setlength{\parskip}{0pt}}
\usepackage{longtable,booktabs,array}
\usepackage{calc} % for calculating minipage widths
% Correct order of tables after \paragraph or \subparagraph
\usepackage{etoolbox}
\makeatletter
\patchcmd\longtable{\par}{\if@noskipsec\mbox{}\fi\par}{}{}
\makeatother
% Allow footnotes in longtable head/foot
\IfFileExists{footnotehyper.sty}{\usepackage{footnotehyper}}{\usepackage{footnote}}
\makesavenoteenv{longtable}

\usepackage{graphicx}
\makeatletter
\newsavebox\pandoc@box
\newcommand*\pandocbounded[1]{% scales image to fit in text height/width
  \sbox\pandoc@box{#1}%
  \Gscale@div\@tempa{\textheight}{\dimexpr\ht\pandoc@box+\dp\pandoc@box\relax}%
  \Gscale@div\@tempb{\linewidth}{\wd\pandoc@box}%
  \ifdim\@tempb\p@<\@tempa\p@\let\@tempa\@tempb\fi% select the smaller of both
  \ifdim\@tempa\p@<\p@\scalebox{\@tempa}{\usebox\pandoc@box}%
  \else\usebox{\pandoc@box}%
  \fi%
}
% Set default figure placement to htbp
\def\fps@figure{htbp}
\makeatother







\usepackage{newtx}

\defaultfontfeatures{Scale=MatchLowercase}
\defaultfontfeatures[\rmfamily]{Ligatures=TeX,Scale=1}





\title{El Comportamiento de la Inflación en el Perú: Un Análisis del
Período 1980-2017}


\shorttitle{El Comportamiento de la Inflación en el Perú}


\usepackage{etoolbox}


\course{Teoría y política monetaria}
\professor{Efrain Castillo Quintero}
\duedate{05/11/2017}

\ccoppy{\textcopyright~2017}



\author{Edison Achalma}



\affiliation{
{Escuela Profesional de Economía, Universidad Nacional de San Cristóbal
de Huamanga}}




\leftheader{Achalma}

\date{2017-05-11}


\abstract{La presente monografía analiza la evolución del fenómeno
inflacionario en el Perú durante el período 1980-2017. Se adopta un
enfoque de compilación académica, sintetizando información histórica,
estadística y teórica para comprender las causas, consecuencias y
dinámica de la inflación en un contexto económico caracterizado por
episodios extremos, como la hiperinflación de finales de los ochenta, y
una posterior estabilización y mantenimiento de tasas bajas. El estudio
se estructura en tres partes fundamentales: (1) una discusión teórica
sobre los conceptos, tipos y causas de la inflación; (2) un análisis
estadístico y narrativo pormenorizado de la tasa de inflación anual,
identificando los factores clave---tanto de política interna como shocks
externos---que explicaron su comportamiento en cada subperíodo; y (3)
una reflexión sobre las implicancias de esta trayectoria para la
economía peruana. Los datos primarios provienen de las memorias anuales
del Banco Central de Reserva del Perú (BCRP) y del Instituto Nacional de
Estadística e Informática (INEI). La principal conclusión es que el Perú
transitó desde una economía con profundos desequilibrios macroeconómicos
e inflación galopante hacia un régimen de estabilidad de precios,
resultado de la adopción de políticas fiscales y monetarias prudentes,
enmarcadas en un esquema de metas explícitas de inflación, que ha
permitido un crecimiento sostenido con baja inflación desde inicios del
siglo XXI. }

\keywords{Inflación, Hiperinflación, Política Monetaria, Estabilidad de
Precios}

\authornote{\par{\addORCIDlink{Edison Achalma}{0000-0001-6996-3364}} 
\par{ }
\par{   El autor no tiene conflictos de interés que revelar.    Los
roles de autor se clasificaron utilizando la taxonomía de roles de
colaborador (CRediT; https://credit.niso.org/) de la siguiente
manera:  Edison Achalma:   conceptualización, metodología, análisis
formal, investigación, recursos, curación de
datos, redacción, visualización, supervisión, administración del
proyecto}
\par{La correspondencia relativa a este artículo debe dirigirse a Edison
Achalma, Escuela Profesional de Economía, Universidad Nacional de San
Cristóbal de Huamanga, Portal Independencia N°
57, Ayacucho, AYA 5001, Perú, Email: \href{mailto:elmer.achalma.09@unsch.edu.pe}{elmer.achalma.09@unsch.edu.pe}}
}

\makeatletter
\let\endoldlt\endlongtable
\def\endlongtable{
\hline
\endoldlt
}
\makeatother

\urlstyle{same}



\makeatletter
\@ifpackageloaded{caption}{}{\usepackage{caption}}
\AtBeginDocument{%
\ifdefined\contentsname
  \renewcommand*\contentsname{Tabla de contenidos}
\else
  \newcommand\contentsname{Tabla de contenidos}
\fi
\ifdefined\listfigurename
  \renewcommand*\listfigurename{Lista de Figuras}
\else
  \newcommand\listfigurename{Lista de Figuras}
\fi
\ifdefined\listtablename
  \renewcommand*\listtablename{Lista de Tablas}
\else
  \newcommand\listtablename{Lista de Tablas}
\fi
\ifdefined\figurename
  \renewcommand*\figurename{Figura}
\else
  \newcommand\figurename{Figura}
\fi
\ifdefined\tablename
  \renewcommand*\tablename{Tabla}
\else
  \newcommand\tablename{Tabla}
\fi
}
\@ifpackageloaded{float}{}{\usepackage{float}}
\floatstyle{ruled}
\@ifundefined{c@chapter}{\newfloat{codelisting}{h}{lop}}{\newfloat{codelisting}{h}{lop}[chapter]}
\floatname{codelisting}{Listado}
\newcommand*\listoflistings{\listof{codelisting}{Lista de Listados}}
\makeatother
\makeatletter
\makeatother
\makeatletter
\@ifpackageloaded{caption}{}{\usepackage{caption}}
\@ifpackageloaded{subcaption}{}{\usepackage{subcaption}}
\makeatother

% From https://tex.stackexchange.com/a/645996/211326
%%% apa7 doesn't want to add appendix section titles in the toc
%%% let's make it do it
\makeatletter
\xpatchcmd{\appendix}
  {\par}
  {\addcontentsline{toc}{section}{\@currentlabelname}\par}
  {}{}
\makeatother

%% Disable longtable counter
%% https://tex.stackexchange.com/a/248395/211326

\usepackage{etoolbox}

\makeatletter
\patchcmd{\LT@caption}
  {\bgroup}
  {\bgroup\global\LTpatch@captiontrue}
  {}{}
\patchcmd{\longtable}
  {\par}
  {\par\global\LTpatch@captionfalse}
  {}{}
\apptocmd{\endlongtable}
  {\ifLTpatch@caption\else\addtocounter{table}{-1}\fi}
  {}{}
\newif\ifLTpatch@caption
\makeatother

\begin{document}

\maketitle


\hypertarget{toc}{}
\tableofcontents
\newpage
\section[Introduction]{El Comportamiento de la Inflación en el Perú}

\setcounter{secnumdepth}{3}

\setlength\LTleft{0pt}




\section{Resumen Ejecutivo}\label{resumen-ejecutivo}

El presente estudio analiza de manera exhaustiva el comportamiento de la
inflación en el Perú durante el período comprendido entre 1980 y 2017,
un lapso que abarca desde la profunda crisis macroeconómica de la década
de los ochenta hasta la consolidación de un régimen de estabilidad de
precios en el siglo XXI. La investigación revela una trayectoria
inflacionaria extraordinaria, caracterizada por un arco que va desde la
hiperinflación más devastadora de la historia del país hasta tasas de
inflación de un dígito, comparables con las de economías desarrolladas.
Este recorrido histórico permite identificar claramente tres grandes
etapas: un primer período de crisis (1980-1990), marcado por la
hiperinflación y el colapso económico; un segundo período de transición
y estabilización (1990-2002), donde se implementaron reformas
estructurales que sentaron las bases para la estabilidad macroeconómica;
y un tercer período de consolidación (2002-2017), en el cual el Perú
adoptó un esquema de Metas Explícitas de Inflación, logrando mantener la
inflación dentro de rangos meta y estableciendo un ancla de credibilidad
para las expectativas económicas. El análisis demuestra que la inflación
peruana no ha sido un fenómeno monolítico, sino que ha estado
profundamente influenciada por factores tanto internos ---como las
políticas fiscales y monetarias, los desequilibrios macroeconómicos y
los choques de oferta internos--- como externos ---como la volatilidad
de los precios internacionales de commodities, los fenómenos climáticos
extremos y las condiciones de la economía global.

El núcleo central del análisis se centra en el período crítico de la
década de 1980, específicamente durante el primer gobierno de Alan
García Pérez (1985-1990), cuando el Perú experimentó su único episodio
de hiperinflación. La investigación detalla cómo un conjunto de
políticas económicas heterodoxas, basadas en el control de precios, el
congelamiento del tipo de cambio y una política fiscal y monetaria
expansiva, desembocaron en un desastre económico de proporciones
históricas. En 1989, la tasa de inflación alcanzó su punto máximo,
superando el \textbf{2,775\% anual}, lo que provocó la destrucción de
dos monedas nacionales (el sol y el inti) y un colapso del producto
interno bruto (PIB) de más del 20\% entre 1988 y 1990. La hiperinflación
empobreció severamente a la población, especialmente a las clases
trabajadoras, y llevó al país a un aislamiento total de los mercados
financieros internacionales. El estudio documenta cómo este período de
``hiper-estanflación'' ---una combinación letal de hiperinflación y
recesión económica--- sirvió como catalizador para un cambio radical en
el modelo económico. La estabilización final llegó con el gobierno de
Alberto Fujimori, quien implementó en 1990 el conocido ``Fujishock'', un
programa de ajuste drástico que incluyó la liberalización del tipo de
cambio, el saneamiento de las finanzas públicas y una estricta
disciplina monetaria, sentando las bases para la erradicación de la
hiperinflación y el inicio de un nuevo ciclo económico.

En contraste con el pasado volátil, el estudio analiza los factores que
han permitido al Perú mantener una inflación controlada en el siglo XXI,
particularmente en el período 2015-2017. Durante estos años, la
inflación anual se situó en niveles históricamente bajos, alcanzando el
\textbf{1.36\% en 2017}, la tasa más baja en ocho años . La
investigación identifica varios elementos clave que explican este
desempeño. En primer lugar, la implementación del esquema de Metas
Explícitas de Inflación por parte del Banco Central de Reserva del Perú
(BCRP) ha sido fundamental para anclar las expectativas inflacionarias y
dotar de credibilidad a la política monetaria. En segundo lugar,
factores de oferta han jugado un papel crucial, como la reversión de los
choques de oferta de años previos (como el Fenómeno del Niño Costero de
2017), que provocaron una caída en los precios de alimentos y energía .
Además, se destaca el papel de la apreciación de la moneda nacional, el
sol, que moderó los efectos de los precios internacionales de
importaciones. Finalmente, el estudio señala que un crecimiento
económico moderado, junto con una demanda interna contenida y la
ausencia de presiones salariales significativas, han contribuido a
mantener las presiones inflacionarias bajo control, permitiendo que la
inflación se mantenga cómodamente dentro del rango meta del BCRP de
\textbf{1\% a 3\%} .

\section{Introducción}\label{introducciuxf3n}

La inflación constituye uno de los fenómenos económicos más relevantes y
complejos de la macroeconomía moderna. Se define como la subida
generalizada y continua del nivel de precios de los bienes y servicios
en una economía durante un período determinado . Este proceso implica
una pérdida del poder adquisitivo de la moneda, ya que cada unidad de
dinero puede comprar una cantidad menor de productos. La medición de la
inflación se realiza principalmente a través del Índice de Precios al
Consumidor (IPC), el cual mide la variación porcentual en el costo de
una canasta de consumo representativa en relación con un período base .
Es importante distinguir la inflación de otros fenómenos como la
deflación, que es la disminución generalizada de precios y que, aunque
pueda parecer beneficiosa para los consumidores en un primer análisis,
suele estar asociada a períodos de crisis económica y recesión, con
graves consecuencias socioeconómicas .

Según el economista Jiménez (2006), la inflación es ``la elevación del
nivel agregado de precios de una economía'' y se mide precisamente a
través de la variación del IPC. Esta definición enfatiza dos
características fundamentales del fenómeno: su carácter generalizado, ya
que no se refiere al aumento de un precio individual sino al conjunto de
precios de la economía, y su persistencia en el tiempo, lo que la
distingue de fluctuaciones puntuales de precios. El IPC, como
instrumento de medición, se construye a partir de una canasta de consumo
representativa que refleja los patrones de gasto de los hogares en un
período base determinado. La inflación implica necesariamente una
pérdida del poder adquisitivo de la moneda, fenómeno que afecta de
manera diferenciada a los distintos agentes económicos, beneficiando a
los deudores (que pagan deudas con dinero de menor valor real) y
perjudicando a los acreedores y a quienes mantienen sus ahorros en
moneda nacional.

El estudio del comportamiento de la inflación en el Perú adquiere una
relevancia particular debido a la extraordinaria volatilidad que ha
caracterizado a su economía en las últimas cuatro décadas. El país ha
experimentado, en un lapso relativamente corto, los extremos del
fenómeno inflacionario: desde la hiperinflación más severa de su
historia, con tasas anuales que superaron el \textbf{7,000\% en 1990},
hasta períodos de deflación, como el registrado en 2001, y una
prolongada etapa de estabilidad de precios con tasas de inflación de un
solo dígito y, en algunos años, incluso inferiores al 2\% . Esta
trayectoria convierte al Perú en un caso de estudio invaluable para
comprender las causas, consecuencias y mecanismos de control de la
inflación.

El análisis de estos períodos permite identificar los errores de
política económica que desencadenaron la crisis, como la falta de
disciplina fiscal, el financiamiento monetario del déficit público y la
implementación de controles de precios ineficaces, así como las reformas
estructurales que permitieron la estabilización y el crecimiento
sostenido. Comprender este proceso no solo tiene un valor histórico,
sino que proporciona lecciones fundamentales para la toma de decisiones
de política pública en el presente y el futuro. El estudio de la
inflación peruana permite evaluar la efectividad de diferentes marcos de
política monetaria, desde el financiamiento fiscal hasta el esquema de
metas de inflación, y entender cómo factores externos e internos
interactúan para determinar la evolución de los precios en una economía
emergente y dependiente de commodities.

\section{Objetivos de la
Investigación}\label{objetivos-de-la-investigaciuxf3n}

\subsection{Objetivo General}\label{objetivo-general}

El objetivo general de la presente investigación es realizar un análisis
exhaustivo y sistemático del comportamiento de la inflación en el Perú
durante el período 1980-2017. Este análisis busca comprender en
profundidad la evolución histórica del fenómeno inflacionario,
identificando los principales períodos de crisis y estabilización, y
explicando los factores económicos, políticos y sociales que
determinaron dicha evolución.

\subsection{Objetivos Específicos}\label{objetivos-especuxedficos}

Para alcanzar el objetivo general, se han establecido los siguientes
objetivos específicos:

\begin{enumerate}
\def\labelenumi{\arabic{enumi}.}
\item
  Explicar los conceptos teóricos fundamentales relacionados con la
  inflación, incluyendo sus diferentes tipos (moderada, galopante,
  hiperinflación y estanflación), sus principales causas (tanto internas
  como externas) y sus consecuencias económicas y sociales. Este
  objetivo busca dotar al lector de las herramientas analíticas
  necesarias para comprender el resto del estudio.
\item
  Presentar y analizar los datos históricos de la tasa de inflación en
  el Perú, desagregando la información por años y por gobiernos. Se
  busca identificar tendencias, ciclos y puntos de inflexión en la serie
  de datos, utilizando fuentes oficiales como el Instituto Nacional de
  Estadística e Informática (INEI) y el Banco Central de Reserva del
  Perú (BCRP).
\item
  Realizar un análisis detallado del comportamiento de la inflación
  durante los diferentes gobiernos que se sucedieron entre 1980 y 2017,
  desde Fernando Belaunde Terry hasta el final del período de análisis.
\item
  Profundizar en el análisis del período crítico de la hiperinflación
  (1985-1990), identificando las causas estructurales y coyunturales que
  llevaron a la crisis, con un enfoque particular en el papel de la
  política fiscal, la deuda externa y las políticas heterodoxas
  implementadas.
\item
  Analizar las reformas económicas implementadas a partir de 1990 para
  controlar la inflación, evaluando su efectividad y los costos sociales
  y económicos asociados. Se busca entender cómo se logró la transición
  de una economía hiperinflacionaria a una de baja y estable inflación.
\end{enumerate}

\section{Discusión Teórica}\label{discusiuxf3n-teuxf3rica}

\subsection{Definición de
Inflación}\label{definiciuxf3n-de-inflaciuxf3n}

La inflación es conceptualizada en la literatura económica como un
fenómeno de carácter monetario que se manifiesta mediante el aumento
sostenido y generalizado del nivel de precios de los bienes y servicios
en una economía. Según la definición proporcionada por Jiménez (2006),
``la inflación es la elevación del nivel agregado de precios de una
economía. Se mide como la variación porcentual en el índice de precios
al consumidor (IPC)'' . Esta definición enfatiza dos características
fundamentales del fenómeno: su carácter generalizado, ya que no se
refiere al aumento de un precio individual sino al conjunto de precios
de la economía, y su persistencia en el tiempo, lo que la distingue de
fluctuaciones puntuales de precios. El IPC, como instrumento de
medición, se construye a partir de una canasta de consumo representativa
que refleja los patrones de gasto de los hogares en un período base
determinado. La inflación implica necesariamente una pérdida del poder
adquisitivo de la moneda, fenómeno que afecta de manera diferenciada a
los distintos agentes económicos, beneficiando a los deudores (que pagan
deudas con dinero de menor valor real) y perjudicando a los acreedores y
a quienes mantienen sus ahorros en moneda nacional.

\subsection{Tipología de la
Inflación}\label{tipologuxeda-de-la-inflaciuxf3n}

La clasificación de la inflación según su magnitud permite distinguir
diferentes grados de severidad del fenómeno, cada uno con
características y consecuencias específicas para la economía.

\subsubsection{Inflación Moderada}\label{inflaciuxf3n-moderada}

La inflación moderada se caracteriza por tasas anuales inferiores al
10\%. En este escenario, los precios suben lentamente y el poder
adquisitivo de los trabajadores no se ve significativamente afectado, ya
que los aumentos salariales pueden compensar, en gran medida, el
incremento en el costo de vida. Este tipo de inflación es considerado
normal y hasta deseable en una economía en crecimiento, ya que puede
facilitar los ajustes económicos sin generar las distorsiones severas
asociadas a niveles más altos de inflación.

\subsubsection{Inflación Galopante}\label{inflaciuxf3n-galopante}

La inflación galopante se presenta cuando la tasa de inflación anual se
sitúa entre el 10\% y el 1000\%. En este contexto, los agentes
económicos comienzan a perder confianza en la moneda y tratan de
desprenderse del dinero lo más rápidamente posible, ya que su valor se
deprecia rápidamente. Se observa una preferencia por mantener activos
reales como bienes raíces, oro o divisas extranjeras, en lugar de dinero
en efectivo. La inflación galopante genera distorsiones significativas
en la economía, afectando la planificación empresarial, desalentando la
inversión a largo plazo y creando incertidumbre sobre los precios
relativos de los bienes y servicios.

\subsubsection{Hiperinflación}\label{hiperinflaciuxf3n}

La hiperinflación es el caso extremo de inflación, con tasas anuales que
superan el 1000\%. En esta situación, el dinero pierde prácticamente
todo su valor y la economía corre el riesgo de volver a un sistema de
trueque o de adoptar una moneda extranjera como medio de cambio. La
hiperinflación está generalmente asociada a conflictos políticos,
guerras o crisis económicas severas que llevan a los gobiernos a
financiar sus gastos mediante la emisión excesiva de papel moneda. En un
escenario de hiperinflación, la economía puede regresar a sistemas de
trueque, ya que el dinero pierde su función como medio de intercambio y
reserva de valor.

\subsubsection{Estanflación}\label{estanflaciuxf3n}

La estanflación es un fenómeno económico complejo que combina la
inflación con un proceso de recesión económica y alto desempleo. Este
escenario rompe con la relación inversa entre inflación y desempleo
postulada por la curva de Phillips, que sostiene que una menor tasa de
desempleo se asocia con una mayor tasa de inflación. En la estanflación,
se presentan simultáneamente una alta inflación y un alto desempleo, lo
que representa un desafío significativo para los responsables de la
política económica. La solución a la estanflación requiere políticas
económicas que actúen sobre la demanda agregada, como el aumento del
gasto público, la reducción de los impuestos o el descenso de las tasas
de interés, con el objetivo de estimular la producción y el empleo sin
agravar la inflación.

\subsection{Causas de la Inflación}\label{causas-de-la-inflaciuxf3n}

Según Le Roy (2001), la inflación ocurre debido a un incremento en la
demanda total, lo que aumenta los precios (inflación por la demanda), o
debido a un incremento en los costos de producción que empuja hacia
arriba los precios de los productos terminados (inflación por alza de
costos). Estas causas pueden clasificarse en internas y externas,
dependiendo de su origen.

\subsubsection{Causas Internas}\label{causas-internas}

Las causas internas de la inflación son aquellas que se originan dentro
de la economía de un país. Entre las más importantes se encuentran:

\begin{itemize}
\tightlist
\item
  \textbf{Emisión excesiva de papel moneda:} Cuando el gobierno o el
  banco central emiten más dinero del que la economía necesita para sus
  transacciones, se produce un exceso de liquidez que presiona al alza
  los precios. Este fue uno de los principales factores de la
  hiperinflación peruana de los años 80, cuando el déficit fiscal se
  financiaba mediante la emisión de moneda.
\item
  \textbf{Oferta insuficiente de productos agropecuarios:} Cuando la
  producción de alimentos no alcanza a cubrir la demanda, los precios de
  estos productos aumentan, lo que tiene un efecto directo en la
  inflación, dado el peso significativo de los alimentos en la canasta
  de consumo.
\item
  \textbf{Excesivo afán de lucro de los capitalistas:} En algunos casos,
  los empresarios pueden aumentar los precios de sus productos para
  obtener mayores ganancias, lo que contribuye a la inflación.
\item
  \textbf{Especulación y acaparamiento de mercancías:} Los especuladores
  pueden acaparar productos para crear escasez artificial y luego
  venderlos a precios más altos, generando presiones inflacionarias.
\item
  \textbf{Altas tasas de interés bancario:} Las altas tasas de interés
  encarecen el crédito, lo que puede aumentar los costos de producción
  y, por tanto, los precios de los productos.
\item
  \textbf{Devaluación:} La devaluación de la moneda local aumenta el
  precio de los productos importados y de los productos nacionales que
  utilizan insumos importados, lo que presiona al alza la inflación.
\end{itemize}

\subsubsection{Causas Externas}\label{causas-externas}

Las causas externas de la inflación provienen del exterior y afectan a
la economía de un país. Entre las más relevantes se encuentran:

\begin{itemize}
\tightlist
\item
  \textbf{Importación excesiva de mercancías a precios muy altos:}
  Cuando un país importa productos a precios elevados, esto puede
  transmitirse a los precios internos, generando inflación.
\item
  \textbf{Afluencia excesiva de capitales externos:} Una gran entrada de
  capitales extranjeros puede aumentar la demanda de la moneda local, lo
  que puede llevar a una apreciación del tipo de cambio y,
  posteriormente, a una devaluación que genere inflación.
\item
  \textbf{Exportación excesiva de algunos productos:} La exportación
  masiva de productos puede reducir su oferta interna, lo que aumenta
  sus precios en el mercado local.
\item
  \textbf{Especulación y acaparamiento a nivel mundial:} Los movimientos
  especulativos en los mercados internacionales de commodities pueden
  provocar aumentos significativos en los precios de los productos
  básicos, lo que afecta a los países importadores.
\item
  \textbf{Excesiva deuda externa:} Una deuda externa muy alta puede
  generar presiones sobre el tipo de cambio y la balanza de pagos, lo
  que puede traducirse en inflación.
\end{itemize}

\subsection{Consecuencias de la
Inflación}\label{consecuencias-de-la-inflaciuxf3n}

La inflación tiene consecuencias económicas y sociales significativas
que afectan a toda la sociedad. Estas consecuencias varían en intensidad
según el nivel de inflación y la estructura económica del país.

\subsubsection{Consecuencias
Económicas}\label{consecuencias-econuxf3micas}

\begin{itemize}
\tightlist
\item
  \textbf{Aumento en el costo de vida:} La inflación reduce el poder
  adquisitivo de los salarios y los ingresos, lo que significa que las
  personas pueden comprar menos bienes y servicios con el mismo monto de
  dinero.
\item
  \textbf{Desincentivo a la inversión a largo plazo:} La incertidumbre
  sobre los precios futuros desalienta la inversión en proyectos de
  largo plazo, ya que los empresarios no pueden predecir con certeza sus
  costos y beneficios.
\item
  \textbf{Beneficio para los deudores:} Los deudores se benefician de la
  inflación, ya que pagan sus deudas con dinero de menor valor real.
\item
  \textbf{Devaluación de la moneda local:} La inflación suele ir
  acompañada de una devaluación de la moneda local, lo que aumenta el
  precio de los productos importados y puede generar una fuga de
  capitales hacia monedas más estables.
\item
  \textbf{Distorsión de los precios relativos:} La inflación,
  especialmente cuando es muy alta, distorsiona los precios relativos de
  los bienes y servicios, lo que lleva a una asignación ineficiente de
  los recursos.
\end{itemize}

\subsubsection{Consecuencias Sociales}\label{consecuencias-sociales}

\begin{itemize}
\tightlist
\item
  \textbf{Empobrecimiento de los sectores de ingresos fijos:} Los
  jubilados, los pensionistas y los trabajadores cuyos salarios no se
  ajustan a la inflación sufren un deterioro en su nivel de vida.
\item
  \textbf{Aumento de la pobreza:} La inflación, especialmente la
  hiperinflación, puede empobrecer a grandes sectores de la población,
  especialmente a los más vulnerables.
\item
  \textbf{Descontento social:} La inflación puede generar descontento
  social y protestas, especialmente cuando afecta de manera
  desproporcionada a los sectores de menores ingresos.
\item
  \textbf{Trueque:} En casos extremos de hiperinflación, el dinero
  pierde su valor como medio de intercambio y la gente recurre al
  trueque de bienes y servicios.
\end{itemize}

\section{Análisis Estadístico del Comportamiento de la Inflación
(1980-2017)}\label{anuxe1lisis-estaduxedstico-del-comportamiento-de-la-inflaciuxf3n-1980-2017}

\subsection{Panorama General de la Evolución de la Tasa de
Inflación}\label{panorama-general-de-la-evoluciuxf3n-de-la-tasa-de-inflaciuxf3n}

La evolución de la tasa de inflación en el Perú entre 1980 y 2017
presenta una trayectoria extraordinaria que refleja las profundas
transformaciones económicas y políticas del país. El período se inicia
con tasas de inflación ya elevadas, producto de los desequilibrios
macroeconómicos de la década anterior. Durante los años 80, la inflación
se mantuvo en niveles de dos y tres dígitos, con un promedio anual que
osciló entre el 60\% y el 111\%, alcanzando su pico más crítico en 1989
durante el gobierno de Alan García. La década de 1990 marca un punto de
inflexión, con la implementación de programas de estabilización que
lograron reducir drásticamente la inflación, de \textbf{7,650\% en 1990}
a niveles de un dígito a partir del año 2000. Esta tendencia descendente
se consolidó en las décadas siguientes, con la inflación manteniéndose
dentro de rangos moderados, aunque con cierta volatilidad explicada por
factores externos como los precios de los commodities y fenómenos
climáticos.

\begin{longtable}[]{@{}
  >{\centering\arraybackslash}p{(\linewidth - 14\tabcolsep) * \real{0.1250}}
  >{\centering\arraybackslash}p{(\linewidth - 14\tabcolsep) * \real{0.1250}}
  >{\centering\arraybackslash}p{(\linewidth - 14\tabcolsep) * \real{0.1250}}
  >{\centering\arraybackslash}p{(\linewidth - 14\tabcolsep) * \real{0.1250}}
  >{\centering\arraybackslash}p{(\linewidth - 14\tabcolsep) * \real{0.1250}}
  >{\centering\arraybackslash}p{(\linewidth - 14\tabcolsep) * \real{0.1250}}
  >{\centering\arraybackslash}p{(\linewidth - 14\tabcolsep) * \real{0.1250}}
  >{\centering\arraybackslash}p{(\linewidth - 14\tabcolsep) * \real{0.1250}}@{}}
\toprule\noalign{}
\begin{minipage}[b]{\linewidth}\centering
Año
\end{minipage} & \begin{minipage}[b]{\linewidth}\centering
Tasa de Inflación (\%)
\end{minipage} & \begin{minipage}[b]{\linewidth}\centering
Año
\end{minipage} & \begin{minipage}[b]{\linewidth}\centering
Tasa de Inflación (\%)
\end{minipage} & \begin{minipage}[b]{\linewidth}\centering
Año
\end{minipage} & \begin{minipage}[b]{\linewidth}\centering
Tasa de Inflación (\%)
\end{minipage} & \begin{minipage}[b]{\linewidth}\centering
Año
\end{minipage} & \begin{minipage}[b]{\linewidth}\centering
Tasa de Inflación (\%)
\end{minipage} \\
\midrule\noalign{}
\endhead
\bottomrule\noalign{}
\endlastfoot
1980 & 59.2 & 1990 & 7,650.0 & 2000 & 3.7 & 2010 & 2.1 \\
1981 & 73.0 & 1991 & 139.2 & 2001 & -0.1 & 2011 & 4.7 \\
1982 & 73.0 & 1992 & 56.7 & 2002 & 1.5 & 2012 & 2.6 \\
1983 & 125.1 & 1993 & 39.5 & 2003 & 2.5 & 2013 & 2.9 \\
1984 & 111.5 & 1994 & 15.4 & 2004 & 3.5 & 2014 & 3.2 \\
1985 & 158.3 & 1995 & 10.2 & 2005 & 1.5 & 2015 & 4.4 \\
1986 & 62.9 & 1996 & 11.8 & 2006 & 1.1 & 2016 & 3.2 \\
1987 & 114.5 & 1997 & 6.5 & 2007 & 3.9 & 2017 & 1.4 \\
1988 & 1,722.3 & 1998 & 6.0 & 2008 & 6.7 & & \\
1989 & 2,775.0 & 1999 & 3.7 & 2009 & 0.2 & & \\
\end{longtable}

\emph{Fuente: Elaboración propia con datos del Banco Central de Reserva
del Perú (BCRP) e Instituto Nacional de Estadística e Informática
(INEI).}

\subsection{Década de 1980: Crisis e
Hiperinflación}\label{duxe9cada-de-1980-crisis-e-hiperinflaciuxf3n}

La década de los ochenta representa uno de los períodos más turbulentos
en la historia económica del Perú, caracterizado por una crisis
económica profunda y una hiperinflación descontrolada. El período
comenzó con la transición a la democracia tras el gobierno militar, con
Fernando Belaunde Terry asumiendo la presidencia en 1980. Durante su
mandato (1980-1985), la inflación se mantuvo en niveles elevados,
oscilando entre el 59\% y el 111\% anual, impulsada por déficits
fiscales persistentes, devaluaciones y la postergación de ajustes en
precios controlados. Sin embargo, fue durante el gobierno de Alan García
Pérez (1985-1990) que la situación se agravó dramáticamente. Las
políticas económicas heterodoxas implementadas, como el control de
precios, los tipos de cambio múltiples y la expansión del crédito
subsidiado, resultaron en un desastre económico. La emisión excesiva de
moneda para financiar el déficit fiscal, sumado a la pérdida de
credibilidad en las instituciones económicas, llevó a una espiral
inflacionaria que culminó en la hiperinflación. En 1988, la inflación
mensual superó el 100\% en septiembre, y en 1990, la inflación anual
alcanzó el \textbf{7,650\%}, destruyendo completamente el valor de la
moneda y empobreciendo a la población. Este período de
``hiperestanflación'' combinó la inflación más alta de la historia del
país con una severa recesión económica, dejando una profunda cicatriz en
la sociedad peruana.

\subsubsection{Gobierno de Fernando Belaunde Terry
(1980-1985)}\label{gobierno-de-fernando-belaunde-terry-1980-1985}

El gobierno de Fernando Belaunde Terry (1980-1985) heredó una economía
con problemas estructurales y una inflación creciente. Durante su
mandato, la tasa de inflación se mantuvo en niveles elevados, oscilando
entre el 59\% y el 111\% anual, impulsada por factores como el déficit
fiscal, la expansión del crédito interno y los desastres naturales como
el Fenómeno del Niño de 1983 . La política de postergación de ajustes en
los precios controlados durante el primer semestre de 1980 solo logró
trasladar presiones inflacionarias a los años siguientes, agravando el
problema. Aunque se observó una breve desaceleración en 1981, la
inflación repuntó con fuerza en los años siguientes, alcanzando tasas de
tres dígitos: 125.1\% en 1983 y 111.5\% en 1984. Estos niveles
reflejaban la incapacidad de las autoridades para controlar los
desequilibrios macroeconómicos subyacentes, como los problemas en las
cuentas externas y públicas, la continua devaluación y las expectativas
inflacionarias arraigadas.

\subsubsection{Gobierno de Alan García Pérez
(1985-1990)}\label{gobierno-de-alan-garcuxeda-puxe9rez-1985-1990}

El gobierno de Alan García Pérez (1985-1990) heredó una economía en
crisis y, aunque inicialmente logró una reducción de la inflación en
1986, la implementación de políticas económicas heterodoxas, como el
control de precios y el anuncio de limitar el pago de la deuda externa
al 10\% de las exportaciones, generó desconfianza y desequilibrios que
agravaron aún más la crisis. La emisión excesiva de moneda para
financiar el déficit fiscal, combinada con la pérdida de valor del inti,
llevó a la economía a un colapso inflacionario sin precedentes. Los años
1988 y 1989 marcaron el pico de la hiperinflación, con tasas anuales de
\textbf{1,722.3\% y 2,775.0\%}, respectivamente . Esta situación se
tradujo en una destrucción casi total del valor adquisitivo de la moneda
nacional, el inti, y obligó al gobierno a implementar medidas drásticas,
como la creación de un nuevo tipo de cambio para importaciones (dólar
MUC) y, finalmente, la sustitución de la moneda. La hiperinflación de
fines de los 80 no solo fue un fenómeno económico, sino también un
síntoma de una profunda crisis política y social que afectó gravemente
el bienestar de la población, especialmente de los sectores más
vulnerables.

\subsection{Década de 1990: Estabilización y
Reducción}\label{duxe9cada-de-1990-estabilizaciuxf3n-y-reducciuxf3n}

La década de los noventa marcó un punto de inflexión en la historia
económica del Perú, con la implementación de un ambicioso programa de
estabilización y reformas estructurales que lograron frenar la
hiperinflación y sentar las bases para el crecimiento económico
sostenido. El proceso de estabilización comenzó en agosto de 1990 con el
gobierno de Alberto Fujimori, quien heredó una economía en caos con una
inflación del 7,650\% anual. El programa de estabilización, conocido
como el ``Fujishock'', incluyó medidas drásticas como la liberalización
del mercado de divisas, la eliminación de subsidios, la privatización de
empresas estatales y la adopción de una política monetaria restrictiva.
Estas medidas, aunque dolorosas en el corto plazo, resultaron efectivas
para reducir la inflación de manera dramática. En 1991, la inflación
cayó al 139\%, y para 1992 ya se encontraba en el 57\%. La coherencia
entre las políticas monetaria y fiscal fue clave para este logro, con el
Banco Central de Reserva del Perú (BCRP) adoptando una política de
estricto control de los agregados monetarios. Durante el resto de la
década, la inflación continuó su tendencia descendente, alcanzando
niveles de un solo dígito por primera vez en décadas. En 1997, la
inflación fue del 6.5\%, y en 1999, del 3.7\%, consolidando el éxito del
programa de estabilización y marcando el inicio de una nueva era de
estabilidad macroeconómica.

\subsubsection{Gobierno de Alberto Fujimori
(1990-2000)}\label{gobierno-de-alberto-fujimori-1990-2000}

El gobierno de Alberto Fujimori (1990-2000) heredó una economía en
bancarrota y una inflación desbocada. En agosto de 1990, se implementó
el ``Fujishock'', un programa de estabilización ortodoxo que incluyó una
masiva corrección de precios (la inflación de agosto de 1990 bordeó el
400\%), la liberalización del tipo de cambio, una severa restricción del
gasto público y una reforma tributaria profunda . Aunque la inflación de
1990 fue la más alta de la historia (7,650\%), estas medidas sentaron
las bases para su control. En 1991, la inflación cayó dramáticamente al
139\%, y continuó su descenso sostenido durante el resto de la década:
56.7\% en 1992, 39.5\% en 1993, 15.4\% en 1994, y finalmente alcanzando
un dígito en 1995 (10.2\%) . Este logro fue posible gracias a la
combinación de una política monetaria restrictiva y la disciplina
fiscal. La independencia del Banco Central de Reserva del Perú,
consagrada en la Constitución de 1993, fue fundamental para consolidar
esta estabilidad y sentar las bases para el crecimiento económico
sostenido con baja inflación.

\subsection{Década de 2000: Consolidación de la Baja
Inflación}\label{duxe9cada-de-2000-consolidaciuxf3n-de-la-baja-inflaciuxf3n}

La década de los dos mil se caracterizó por la consolidación de la
estabilidad de precios en el Perú, con la implementación del esquema de
Metas Explícitas de Inflación en 2002 como el pilar central de la
política monetaria. Este marco institucional, que establece un rango
meta de inflación de 1\% a 3\%, ha sido fundamental para anclar las
expectativas de los agentes económicos y mantener la inflación bajo
control. Durante este período, el Perú logró mantener una de las tasas
de inflación más bajas y estables de América Latina, con un promedio
anual de 2.5\% entre 2002 y 2005, una situación que no se presentaba
desde la década de 1960. Sin embargo, la década no estuvo exenta de
desafíos. La volatilidad de los precios internacionales de los
commodities, especialmente el petróleo y los alimentos, generó presiones
inflacionarias puntuales. En 2008, la inflación se aceleró hasta el
6.65\%, pero la política monetaria del BCRP logró contener el impacto y
mantener la inflación dentro del rango meta en la mayoría de los años.
El crecimiento económico sostenido, el superávit fiscal y el
fortalecimiento de las instituciones económicas también contribuyeron a
este entorno de estabilidad de precios.

La década de 2000 se caracterizó por la consolidación de la baja
inflación y la implementación del régimen de Metas Explícitas de
Inflación. En 2001, el Perú experimentó un episodio de deflación
(-0.13\%), el primero desde 1939, debido a la reducción de precios de
combustibles y servicios públicos. Sin embargo, a partir de 2002, con la
implementación del esquema de metas de inflación, la tasa de inflación
se estabilizó en niveles bajos y controlados: 1.52\% en 2002, 2.48\% en
2003, 3.48\% en 2004, 1.5\% en 2005, 2.0\% en 2006, 1.1\% en 2007,
6.65\% en 2008 (afectada por la crisis alimentaria mundial), 0.25\% en
2009 y 2.08\% en 2010. El promedio anual de inflación durante esta
década fue de 2.5\%, la más baja desde la década de 1960. Este logro fue
posible gracias a la credibilidad ganada por el BCRP en la
implementación de su política monetaria, la disciplina fiscal mantenida
por el gobierno y un contexto internacional favorable de crecimiento
económico y precios estables de commodities.

\subsection{Período 2010-2017: Metas de Inflación y
Volatilidad}\label{peruxedodo-2010-2017-metas-de-inflaciuxf3n-y-volatilidad}

Durante el período 2010-2017, el Perú continuó con su régimen de metas
de inflación, logrando mantener tasas de inflación bajas y estables,
aunque con algunas fluctuaciones debido a factores externos. En 2010, la
tasa de inflación fue de 2.08\%, influenciada por el aumento de las
cotizaciones internacionales de algunos alimentos y combustibles, así
como por condiciones climatológicas adversas que afectaron los precios
de algunos productos. En 2011, la inflación aumentó a 4.74\%, explicada
tanto por choques externos (aumento de precios internacionales de
commodities) como internos (anomalías climatológicas que afectaron la
oferta de productos agrícolas). Sin embargo, en los años siguientes, la
inflación se redujo gradualmente, ubicándose en 2.65\% en 2012 y 2.86\%
en 2013. En 2014, la inflación fue de 3.22\%, y en 2015, se aceleró a
4.4\%, reflejando principalmente alzas en los precios de alimentos,
tarifas eléctricas y de los rubros vinculados al tipo de cambio. A pesar
de estas fluctuaciones, la inflación se mantuvo dentro del rango meta
establecido por el BCRP, que para este período era de 1\% a 3\%. La
capacidad del BCRP para mantener la inflación baja y estable, a pesar de
los choques de oferta externos e internos, demostró la efectividad del
régimen de metas de inflación y la credibilidad de la política
monetaria. En el período 2003-2012, la tasa de inflación anual promedio
fue de 2.9\%, la más baja de América Latina, lo que consolidó al Perú
como un país con una de las mejores trayectorias de estabilidad de
precios en la región.

\subsubsection{Período de Metas de Inflación
(2011-2017)}\label{peruxedodo-de-metas-de-inflaciuxf3n-2011-2017}

Durante el período 2011-2017, el Perú continuó operando bajo el esquema
de Metas Explícitas de Inflación, aunque con cierta volatilidad. En
2011, la inflación se aceleró a 4.74\% debido al alza de precios
internacionales de commodities y anomalías climáticas que afectaron la
oferta de alimentos . En los años siguientes, la inflación se moderó,
ubicándose en 2.6\% en 2012, 2.9\% en 2013, 3.2\% en 2014 y 4.4\% en
2015. En 2016, la tasa fue de 3.2\%, y en 2017 cerró en un bajo 1.4\%,
el nivel más bajo desde 2009 . Este resultado se explicó por la
reversión de choques de oferta en productos agrícolas y condiciones
climáticas favorables. El BCRP desempeñó un papel crucial durante este
período, utilizando la tasa de interés de referencia y otras
herramientas de política monetaria para mantener la inflación anclada a
la meta, demostrando la madurez del marco de política monetaria
implementado.

\section{Conclusiones}\label{conclusiones}

El análisis del comportamiento de la inflación en el Perú durante el
período 1980-2017 permite identificar tres grandes etapas claramente
diferenciadas. La primera, de 1980 a 1990, se caracterizó por una crisis
inflacionaria de proporciones históricas, con tasas de hiperinflación
que alcanzaron el \textbf{7,650\% anual en 1990}, destruyendo el valor
de la moneda y empobreciendo a la población. La segunda etapa, de 1991 a
2002, fue un período de transición y estabilización, donde la
implementación de reformas estructurales y políticas monetarias y
fiscales más disciplinadas permitió reducir drásticamente la inflación.
Finalmente, la tercera etapa, de 2003 a 2017, se caracterizó por la
consolidación de la baja inflación, gracias a la implementación del
esquema de Metas Explícitas de Inflación, que ha permitido mantener la
inflación dentro de rangos moderados y estables.

Los factores determinantes de la inflación peruana han sido tanto
internos como externos. Entre los factores internos, destacan la
política fiscal expansiva, el financiamiento monetario del déficit
público, la devaluación de la moneda y los choques de oferta en el
sector agrícola. Entre los factores externos, la volatilidad de los
precios internacionales de los commodities, especialmente el petróleo y
los alimentos, ha sido un determinante clave de la inflación en el Perú.
La experiencia del país demuestra que la credibilidad de las
instituciones económicas, especialmente del Banco Central, es
fundamental para anclar las expectativas inflacionarias y mantener la
estabilidad de precios.

\section{Recomendaciones}\label{recomendaciones}

Con base en el análisis realizado, se proponen las siguientes
recomendaciones de política económica para mantener la estabilidad de
precios en el Perú:

\begin{enumerate}
\def\labelenumi{\arabic{enumi}.}
\tightlist
\item
  \textbf{Mantener la disciplina fiscal:} Es fundamental que el gobierno
  mantenga un equilibrio entre sus ingresos y gastos, evitando el
  financiamiento del déficit público mediante la emisión de moneda. Una
  política fiscal prudente es un pilar fundamental para la estabilidad
  macroeconómica.
\item
  \textbf{Fortalecer la independencia del Banco Central:} La autonomía
  del BCRP debe ser preservada y fortalecida, para que pueda implementar
  una política monetaria coherente y efectiva, sin presiones políticas
  de corto plazo.
\item
  \textbf{Continuar con el esquema de Metas Explícitas de Inflación:} El
  régimen de metas de inflación ha demostrado ser efectivo para mantener
  la inflación bajo control. Es importante continuar con este esquema,
  asegurando la transparencia y la comunicación clara de las decisiones
  de política monetaria.
\item
  \textbf{Desarrollar el mercado interno de alimentos:} Para reducir la
  vulnerabilidad de la inflación a los choques de oferta agrícola, es
  necesario desarrollar el mercado interno de alimentos, mejorando la
  infraestructura de transporte y almacenamiento, y promoviendo la
  competencia en el sector.
\item
  \textbf{Diversificar la economía:} La dependencia del Perú de las
  exportaciones de commodities lo hace vulnerable a las fluctuaciones de
  los precios internacionales. Es necesario diversificar la economía,
  promoviendo el desarrollo de sectores productivos con mayor valor
  agregado.
\end{enumerate}

\section{Perspectivas Futuras}\label{perspectivas-futuras}

Las perspectivas futuras para la inflación en el Perú dependen de una
combinación de factores internos y externos. En el plano interno, la
consolidación de las instituciones económicas y la continuidad de
políticas fiscales y monetarias prudentes son fundamentales para
mantener la estabilidad de precios. El fortalecimiento del marco de
política monetaria y la mejora en la coordinación entre las políticas
fiscal y monetaria serán claves para enfrentar los desafíos futuros.

En el plano externo, la volatilidad de los precios de los commodities
seguirá siendo un factor de riesgo para la inflación peruana. El cambio
climático y la mayor frecuencia de fenómenos climáticos extremos podrían
afectar la oferta de alimentos y generar presiones inflacionarias.
Asimismo, las condiciones de la economía global, incluyendo las tasas de
interés de los países desarrollados y la demanda de los principales
socios comerciales del Perú, también influirán en la evolución de la
inflación.

\section{Referencias
Bibliográficas}\label{referencias-bibliogruxe1ficas}

\begin{itemize}
\tightlist
\item
  Banco Central de Reserva del Perú. (s.f.). \emph{Memorias}. Recuperado
  de http://www.bcrp.gob.pe/publicaciones/memoria-anual.html
\item
  Jiménez, F. (2006). \emph{Macroeconomía. Enfoques y modelos. (Tomo
  I)}. Lima: Pontificia Universidad Católica del Perú.
\item
  Le Roy, M. R. (2001). \emph{Macroeconomía Moderna} (7ma. Ed.). México:
  Mc Graw-Hill.
\end{itemize}

\appendix

\section{Anexos}\label{apx-anexos}

\subsection{Anexo Estadístico: Tablas de Datos de
Inflación}\label{anexo-estaduxedstico-tablas-de-datos-de-inflaciuxf3n}

\begin{longtable}[]{@{}lll@{}}
\toprule\noalign{}
Año & Inflación Anual (\%) & Gobierno de Turno \\
\midrule\noalign{}
\endhead
\bottomrule\noalign{}
\endlastfoot
1980 & 59.2 & Fernando Belaunde Terry \\
1981 & 73.0 & Fernando Belaunde Terry \\
1982 & 73.0 & Fernando Belaunde Terry \\
1983 & 125.1 & Fernando Belaunde Terry \\
1984 & 111.5 & Fernando Belaunde Terry \\
1985 & 158.3 & Alan García Pérez \\
1986 & 62.9 & Alan García Pérez \\
1987 & 114.5 & Alan García Pérez \\
1988 & 1,722.3 & Alan García Pérez \\
1989 & 2,775.0 & Alan García Pérez \\
1990 & 7,650.0 & Alan García Pérez / Alberto Fujimori \\
1991 & 139.2 & Alberto Fujimori \\
1992 & 56.7 & Alberto Fujimori \\
1993 & 39.5 & Alberto Fujimori \\
1994 & 15.4 & Alberto Fujimori \\
1995 & 10.2 & Alberto Fujimori \\
1996 & 11.8 & Alberto Fujimori \\
1997 & 6.5 & Alberto Fujimori \\
1998 & 6.0 & Alberto Fujimori \\
1999 & 3.7 & Alberto Fujimori \\
2000 & 3.7 & Alberto Fujimori \\
2001 & -0.1 & Valentín Paniagua / Alejandro Toledo \\
2002 & 1.5 & Alejandro Toledo \\
2003 & 2.5 & Alejandro Toledo \\
2004 & 3.5 & Alejandro Toledo \\
2005 & 1.5 & Alejandro Toledo \\
2006 & 1.1 & Alejandro Toledo \\
2007 & 3.9 & Alan García Pérez \\
2008 & 6.7 & Alan García Pérez \\
2009 & 0.2 & Alan García Pérez \\
2010 & 2.1 & Alan García Pérez \\
2011 & 4.7 & Ollanta Humala \\
2012 & 2.6 & Ollanta Humala \\
2013 & 2.9 & Ollanta Humala \\
2014 & 3.2 & Ollanta Humala \\
2015 & 4.4 & Ollanta Humala \\
2016 & 3.2 & Pedro Pablo Kuczynski \\
2017 & 1.4 & Pedro Pablo Kuczynski \\
\end{longtable}

\emph{Fuente: Elaboración propia con datos del BCRP e INEI.}

\subsection{Anexo Metodológico: Fuentes de Información y Metodología de
Cálculo del
IPC}\label{anexo-metodoluxf3gico-fuentes-de-informaciuxf3n-y-metodologuxeda-de-cuxe1lculo-del-ipc}

\subsubsection{Fuentes de Información}\label{fuentes-de-informaciuxf3n}

Los datos estadísticos utilizados en esta monografía provienen de
fuentes oficiales y confiables, principalmente del \textbf{Banco Central
de Reserva del Perú (BCRP)} y del \textbf{Instituto Nacional de
Estadística e Informática (INEI)} . Estas instituciones son responsables
de la elaboración y publicación de las estadísticas económicas del país,
incluyendo el Índice de Precios al Consumidor (IPC), que es el indicador
oficial de medición de la inflación.

\subsubsection{Metodología de Cálculo del
IPC}\label{metodologuxeda-de-cuxe1lculo-del-ipc}

El IPC es un indicador estadístico que mide la variación promedio de los
precios de una canasta de bienes y servicios representativa del consumo
de los hogares. La metodología de cálculo del IPC en el Perú se basa en
las siguientes etapas:

\begin{enumerate}
\def\labelenumi{\arabic{enumi}.}
\tightlist
\item
  \textbf{Selección de la canasta de consumo:} Se selecciona una canasta
  de bienes y servicios que represente el patrón de consumo de los
  hogares en Lima Metropolitana. Esta canasta se actualiza
  periódicamente para reflejar los cambios en los hábitos de consumo.
\item
  \textbf{Asignación de ponderaciones:} A cada producto de la canasta se
  le asigna un peso o ponderación que refleja su importancia en el gasto
  total de los hogares. Los productos con mayor ponderación tienen un
  mayor impacto en el IPC.
\item
  \textbf{Recolección de precios:} Se recopilan los precios de los
  productos de la canasta en una muestra de establecimientos comerciales
  de Lima Metropolitana. Esta recolección se realiza de manera regular,
  generalmente semanal o mensual.
\item
  \textbf{Cálculo del índice:} Se calcula el índice para cada producto y
  para el conjunto de la canasta, utilizando una fórmula que considera
  la variación de los precios y las ponderaciones de cada producto. El
  IPC se publica mensualmente y anualmente.
\end{enumerate}






\end{document}
